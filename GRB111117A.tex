%                                                                 aa.dem
% AA vers. 8.2, LaTeX class for Astronomy & Astrophysics
% demonstration file
%                                                       (c) EDP Sciences
%-----------------------------------------------------------------------
%
%\documentclass[referee]{aa} % for a referee version
%\documentclass[onecolumn]{aa} % for a paper on 1 column  
%\documentclass[longauth]{aa} % for the long lists of affiliations 
%\documentclass[rnote]{aa} % for the research notes
%\documentclass[letter]{aa} % for the letters 
%\documentclass[bibyear]{aa} % if the references are not structured 
% according to the author-year natbib style

%%%%%%%%%%%%%%%%%%%%%%%%%%%%%%%%%%%%%%%%%%%%%%%%%%%%%%%%%%%%%%%%%%%%%%%%%%%%
\documentclass{aa}    %% Astronomy & Astrophysics style class aa.cls v8.2
%\documentclass[referee]{aa} 

\usepackage{graphicx,url,twoopt,natbib}
\usepackage[varg]{txfonts}           %% A&A font choice

\usepackage{pdfcomment}              %% for popup acronym meanings
\usepackage{acronym}                 %% for popup acronym meanings

\usepackage{url}
\usepackage{color,hyperref}
\definecolor{linkcolor}{rgb}{0,0.3,0.7}
\hypersetup{colorlinks=true,
	linkcolor=linkcolor, 
	citecolor=linkcolor,
	filecolor=linkcolor, 
	urlcolor=linkcolor}

\usepackage{natbib}

\bibpunct{(}{)}{,}{a}{}{,}    %% natbib cite format used by A&A and ApJ

% Manually specified definitions
\newcommand{\farc}{\hbox{$.\!\!^{\prime\prime}$}} 
\newcommand{\ergA}{$\rm{erg\,cm^{-2}\,s^{-1}\,\AA^{-1}}$} 
\newcommand{\erg}{$\rm{erg\,cm^{-2}\,s^{-1}}$} 
\newcommand{\kms}{$\rm{km\,s^{-1}}\,$}

\newcommand{\griz}{$g' r' i' z'$}
\newcommand{\JHK}{$JHK_{\rm{s}}$}
\newcommand{\gK}{$g' r' i' z' JHK_{\rm{s}}$}
\newcommand{\Msun}{$M_\odot$}


% Elements
\newcommand{\lya}{Ly$\alpha$} 
\newcommand{\lyb}{Ly$\beta$} 
\newcommand{\lyg}{Ly$\gamma$} 
\newcommand{\hb}{H$\beta$} 
\newcommand{\ha}{H$\alpha$} 
\newcommand{\hg}{H$\gamma$} 
\newcommand{\hi}{\ion{H}{i}} 
\newcommand{\hei}{[\ion{He}{i}]} 
\newcommand{\oi}{[\ion{O}{i}]} 
\newcommand{\sii}{[\ion{S}{ii}]} 
\newcommand{\siii}{[\ion{S}{iii}]} 
\newcommand{\oii}{[\ion{O}{ii}]} 
\newcommand{\oiii}{[\ion{O}{iii}]}
\newcommand{\nii}{[\ion{N}{ii}]} 
\newcommand{\nv}{[\ion{N}{v}]} 
\newcommand{\neiii}{[\ion{Ne}{iii}]} 
\newcommand{\NIii}{[\ion{Ni}{ii}]} 
\newcommand{\feii}{\ion{Fe}{ii}} 
\newcommand{\civ}{\ion{C}{iv}} 
\newcommand{\cii}{\ion{C}{ii}}
\newcommand{\mgi}{\ion{Mg}{i}} 
\newcommand{\mgii}{\ion{Mg}{ii}} 
\newcommand{\ali}{\ion{Al}{i}} 
\newcommand{\alii}{\ion{Al}{ii}} 
\newcommand{\aliii}{\ion{Al}{iii}} 
\newcommand{\SIi}{\ion{Si}{i}} 
\newcommand{\SIii}{\ion{Si}{ii}} 
\newcommand{\SIiii}{\ion{Si}{iii}} 
\newcommand{\SIiv}{\ion{Si}{iv}} 
\newcommand{\znii}{\ion{Zn}{ii}} 
\newcommand{\crii}{\ion{Cr}{ii}} 
\newcommand{\mnii}{\ion{Mn}{ii}} 
\newcommand{\tiii}{\ion{Ti}{ii}} 
\newcommand{\caii}{\ion{Ca}{ii}} 


\usepackage{xcolor}
\newcommand\todo[1]{\textbf{(#1)}}

\begin{document}
	
\title{The host galaxy of the short GRB~111117A at $z = 2.211$\thanks{Based on observations collected at the European Southern Observatory, Paranal, Chile, Program ID: 088.A-0051 and 091.D-0904.}}

\titlerunning{GRB~111117A}


\author{J.~Selsing\inst{1}
\and T.~Kr\"{u}hler\inst{2}
\and D.~Malesani\inst{1,3}
\and P.~D'Avanzo\inst{4}
\and J.~Palmerio\inst{5}
\and S.~D.~Vergani\inst{6}
\and J.~Japelj\inst{7}
\and B.~Milvang-Jensen\inst{1}
\and D.~Watson\inst{1}
\and P.~Jakobsson\inst{8}
%
\and J.~Bolmer \inst{2}
\and Z.~Cano\inst{9}
\and S.~Covino \inst{10}
%\and L.~B.~Christensen\inst{1}
\and V.~D'Elia\inst{10}
\and A.~de~Ugarte~Postigo\inst{9, 1}
\and J.~P.~U.~Fynbo\inst{1}
\and A.~Gomboc\inst{11}
\and K.~E.~Heintz\inst{8,1}
\and A.~J.~Levan \inst{12}
\and S.~Piranomonte \inst{13}
\and V.~Pugliese \inst{7}
\and R.~S\'{a}nchez-Ram\'{\i}rez \inst{9, 10}
\and S.~Schulze \inst{14, 15}
\and M.~Sparre\inst{1,16}
\and N.~R.~Tanvir\inst{17}
\and C.~C.~Th\"one\inst{9}
\and K.~Wiersema \inst{17}
}


\institute{Dark Cosmology Centre, Niels Bohr Institute, University of Copenhagen, Juliane Maries Vej 30, 2100 K\o benhavn \O, Denmark
%2
\and Max-Planck-Institut f\"{u}r extraterrestrische Physik, Giessenbachstra\ss e, 85748 Garching, Germany
%3
\and DTU Space, National Space Institute, Technical University of Denmark, Elektrovej 327, DK-2800 Lyngby, Denmark
%4
\and INAF - Osservatorio Astronomico di Brera, via E. Bianchi 46, I-23807, Merate (LC), Italy
%5
\and Sorbonne Universités, UPMC Univ. Paris 6 et CNRS, UMR 7095, Institut d’Astrophysique de Paris, 98 bis bd Arago, 75014 Paris, France
%6
\and GEPI, Observatoire de Paris, PSL Research University, CNRS, Univ. Paris Diderot, Sorbonne Paris Cité, 5 Place Jules Janssen, 92195 Meudon, France
%7
\and Anton Pannekoek Institute for Astronomy, University of Amsterdam, Science Park 904, 1098 XH Amsterdam, The Netherlands
%8
\and Centre for Astrophysics and Cosmology, Science Institute, University of Iceland, Dunhagi 5, 107 Reykjav\'ik, Iceland
%9
\and Instituto de Astrof\'isica de Andaluc\'ia (IAA-CSIC), Glorieta de la Astronom\'ia s/n, E-18008, Granada, Spain.
%10
\and INAF-Osservatorio Astronomico di Roma, Via Frascati 33, I-00040 Monteporzio Catone, Italy;
ASI-Science Data Centre, Via del Politecnico snc, I-00133 Rome, Italy
%11
\and Centre for Astrophysics and Cosmology, University of Nova Gorica, Vipavska 13, 5000 Nova Gorica, Slovenia.
%12
\and Department of Physics, University of Warwick, Coventry CV4 7AL, UK
%13
\and INAF, Osservatorio Astronomico di Brera, Via E. Bianchi 46, I-23807 Merate (LC), Italy
%14
\and Instituto de Astrof\'isica, Facultad de F\'isica, Pontificia Universidad Cat\'olica de Chile, Vicu\~{n}a Mackenna 4860, 7820436 Macul, Santiago, Chile
%15
\and Millennium Institute of Astrophysics, Vicu\~{n}a Mackenna 4860, 7820436 Macul, Santiago, Chile
%16
\and Heidelberger Institut f{\"u}r Theoretische Studien, Schloss-Wolfsbrunnenweg 35, 69118 Heidelberg, Germany 
%17
\and Department of Physics and Astronomy, University of Leicester, University Road, Leicester, LE1 7RH, UK
%\and European Southern Observatory, Alonso de C\'{o}rdova 3107, Vitacura, Casilla 19001, Santiago 19, Chile
}


\date{Received/ accepted}


\authorrunning{Selsing et al.}


\abstract{
It is notoriously difficult to localize short $\gamma$-ray bursts (sGRBs) and
their hosts to measure their redshifts. These measurements, however, are
critical to constrain sGRB progenitors and delay time models. Here, we present
spectroscopy of the host galaxy of GRB~111117A and measure its redshift to be $z
= 2.211$. This makes GRB~111117A the most distant high-confidence short GRB
detected to date. Our spectroscopic redshift supersedes a lower redshift value
for this burst previously estimated from photometry.

We use the spectroscopic redshift, as well as new imaging data to constrain the
nature of the host galaxy as well as physical parameters of the GRB. The rest
frame X-ray derived hydrogen column density, for example, is high compared to a
complete sample of sGRBs and seems to follow the evolution with redshift as
traced by the hosts of long GRBs (lGRBs). This is consistent with the proportion
of the sGRB population originating in late-type galaxies live in similar
environments to that of lGRB hosts.

The host lies in the brighter end of the expected host brightness distribution
at $z = 2.211$, and is actively forming stars. Using the host as a benchmark for
redshift determination, we find that less than 55 per cent of all sGRB redshifts
should be missed due to host faintness at $z\sim2$. The high redshift of
GRB~111117A is evidence against a lognormal delay-time model for sGRBs through
the predicted redshift distribution of sGRBs, which is very sensitive to
high-$z$ sGRBs.

From the age of the universe at explosion time, an initial progenitor separation
of $a_0 < 3.2~R_\odot$ is required in the case of a binary neutron star system. 
This puts constraints on the progenitor system evolution up to the time
of explosion.
}

\keywords{Gamma-ray burst: individual: GRB~111117A --- }

\maketitle

%%%%%%%%%%%%%%%%%%%%%%%%%%%%%%%%%%%%%%%%%%%%%%%%%%%%%%%%%%%%%%%%%%%%%%%%%%%%
\section{Introduction}


There is now mounting evidence that most short-duration gamma-ray bursts (GRBs)
come from the merger of neutron stars (NSs), either with another NS, or a black
hole, due to their apparent association with kilonovae \citep{Barnes2013a,
	Tanvir2013b, Yang2015, Jin2016, Rosswog2016}. The absence of associated
supernovae in deep searches \citep[e.g.][]{Hjorth2005a,Fox2005,Hjorth2005b}
supports this idea and distinguishes the physical origin of sGRBs from their
long-duration counterparts. %\citep[albeit see also][]{Fynbo2006b, Valle2006, Gal-Yam2006}.

The host galaxies of short GRBs are diverse. They are more massive and less actively 
star-forming on average than long GRB hosts \citep{Fong2013b}, while in some cases, no host 
galaxy can be identified \citep{Berger2010a, Tunnicliffe2014}. Together with their position 
within their hosts \citep{Fong2013a}, this suggests a progenitor system that can be very
long-lived, and is associated with stellar mass rather than star-formation rate
\citep{Berger2014}. The median redshift for sGRBs is $z\sim0.5$ \citep{Berger2014}, and
because most of these measurements come from the associated hosts, it is arguably biased
towards lower redshifts.

%So far, the electromagnetic signal from sGRBs are the only means to accurately
% localize NS mergers and holds the promise for detection of an associated
% gravitational wave signal \citep{Ghirlanda2016}. From the burst and following
% afterglow, modeling the properties and mapping the environments additionally
% allows insights into the phenomenon itself.

The total lifetimes of NS binaries depends on their initial separations and
subsequent inspiral times, and impacts the timing and distribution of the
enrichment of the ISM and subsequent stars and planets with heavy $r$-process
elements \citep{VandeVoort2015, Wallner2015,  Ji2016}. Some limits can be
calculated based on models of star-formation histories of, and the spatial
distribution of sGRBs in, their host galaxies \citep[][]{Berger2014}. The most
distant cosmological bursts, however, offer direct, hard limits.


In this \emph{Letter} we present the spectrum of the host galaxy of the short GRB~111117A ($T_{90}=0.46$~s) and measure its redshift to be $z=2.211$. This value is significantly higher 
than the previously estimated redshift based on photometric studies. 
We present the GRB's rest frame 
properties based on this new distance compared to previous analyses
\citep{Margutti2012,Sakamoto2013} and revisit the host properties derived from
the new solution to the SED fit. %While no optical afterglow was detected, the
% excellent localization from a detection
%of the X-ray afterglow by the \emph{Chandra X-ray Observatory} also allows
%us to discuss the positioning and environmental properties of this remarkably
%distant short GRB.

Throughout this \emph{Letter} we use the $\Lambda$CDM cosmology provided by
\citet{Planck2015} in which the universe is flat with $H_0 = 67.7$\,km\,s$^{-1}$
and $\Omega_m = 0.307$ and report all magnitudes in the AB system.

%%%%%%%%%%%%%%%%%%%%%%%%%%%%%%%%%%%%%%%%%%%%%%%%%%%%%%%%%%%%%%%%%%%%%%%%%%%

%%%%%%%%%%%%%%%%%%%%%%%%%%%%%%%%%%%%%%%%%%%%%%%%%%%%%%%%%%%%%%%%%%%%%%%%%%%
\section{Observations and results}
%%%%%%%%%%%%%%%%%%%%%%%%%%%%%%%%%%%%%%%%%%%%%%%%%%%%%%%%%%%%%%%%%%%%%%%%%%%

\subsection{Spectroscopic observations and analysis}

\begin{figure}
	\centering
	\includegraphics[width=9cm]{figures/GRB111117A_spec_obs2.pdf}
	\caption{Imaging of the field of GRB~111117A with the X-shooter slit overlaid. Only one slit is shown, despite four epochs of spectroscopic observations because of the similarity in position angle. The image is the FORS2 $R$-band image for which the photometry is shown in Fig.~\ref{fig:SED}. 
	The blue asterisk indicates the GRB position as derived from the \emph{Chandra} observations in \citet{Sakamoto2013}. The insert shows the 2D-image of the \oiii$\lambda$5007 emission line. The location of a bright skyline is marked by the blue box. The location of the emission line is indicated with the red ellipse. Because the host is observed in nodding-mode, negative images of the emission line appear on both sides in the spatial direction.}
	\label{fig:spec_setup}
\end{figure}


Spectroscopic observations were carried out using the cross-dispersed echelle
spectrograph, VLT/X-shooter \citep{Vernet2011}, at four seperate epochs. The
burst was followed up 38 hours after the Burst Alert Telescope (BAT) trigger
under ESO programme 088.A-0051 (PI: Fynbo) and again later using a different ESO
program 091.D-0904 (PI: Hjorth). X-shooter can cover the wavelength range from
3000~\AA~to 24\,800~\AA~(21\,000~\AA~when the $K$-band blocking filter is used)
across three spectroscopic arms. The bias-correction, flat-fielding, order
tracing, wavelength calibration, rectification, and flux calibration is then
carried out using the VLT/X-shooter pipeline version 2.8.4
\citep{Modigliani2010}. %Because the spectra are curved across each detector, a
% rectification algorithm
%is employed which introduces correlations between neighboring pixels. We select
%a pixel-scale of 0.2/0.2/0.6 \AA/pix for the UVB/VIS/NIR arm to minimize the
%degree of correlation while conserving the maximal resolution.
The observations are combined and extracted using scripts described in Selsing
et al. 2017 (in prep.) and available
online\footnote{\url{https://github.com/jselsing/XSGRB_reduction_scripts}}. The
signal-to-noise of the continuum in the near-infrared arm is too low to use the
optimal extraction algorithm \citep{Horne1986} and therefore the extraction is
carried out with a simple aperture. An overview of the spectroscopic
observations is given in Table~\ref{tab:spec_overview}, and the position of the
slit on the target is shown in Fig.~\ref{fig:spec_setup}.We show the extracted
spectrum in Fig.~\ref{fig:SED}.

We determine a redshift of $z = 2.211$ from the simultaneous detection of
emission lines interpreted as \lya, \oii, \hb, \oiii$\lambda$5007, and \ha, with
\hb~detected at low significance ($\sim 3 \sigma$). We show \oiii$\lambda$5007
in the insert in Fig.~\ref{fig:spec_setup}.  \ha~is only visible in the first
epoch, due to the $K$-band blocking filter used for the remainder observations.
The nebular lines exhibit a spatial extent of $\sim$ 1\farc5 and show
significant velocity structure along the slit. A drop in the continuum to the
blue of the \lya~line further supports the inferred redshift.

%The total extent of the lines in velocity space is $\sim$ 450 km/s. The line
% profiles shows an asymmetric "double-horned" profile, indicating that we are
% seeing a galaxy with a large degree of coherent rotational motion relative to
% the line-of-sight. If we assume that we are viewing a spiral galaxy exactly
% edge-on, this is a measure of the rotational velocity of the gas. If we assume
% that the spectral resolution and the turbulent width of the lines are
% negligible compared to the rotational velocity, we can based on the projected
% size of the source and the width of the lines, put a constraint on the
% dynamical mass of the galaxy. Based on the physical scale from extent and the
% velocity with, we infer M$_\text{dyn} \gtrsim 10^{10.8}$ M$_\odot$. Because we
% are viewing the host at an angle, this value is an upper limit, which should
% be divided by $\sin(i)$, where $i$ is the inclination relative to edge-on.

Using the luminosity of \ha, we can infer the star-formation rate (SFR) of a 
galaxy \citep{Kennicutt1998}. At the redshift of the GRB host, \ha~is observed at 
21\,000~\AA~where the night sky is very bright. In addition, several bright sky-lines 
intersect the line, making an accurate estimate of the \ha-flux difficult. 
We obtain a limit on the SFR by integrating the part of \ha~free of contamination and 
correcting for the missing flux using the line shape. After converting the 
\citet{Kennicutt1998} relation to a \citet{Chabrier2003} initial mass function using
\citet{Madau2014}, we derive a limit of $SFR > 7 M_\odot$~yr$^{-1}$. 
From the SED-fit (Sect.~\ref{SED}), and the detection of \lya, the host is
constrained to contain very little or no dust, which is why we do not apply 
a dust-correction to the measured \ha~flux here. 

%Secondly, based on the width of \oiii, an aperture covering the line
% is integrated over, where synthetic sky lines \citep{Noll2012, Jones2013} have
% been masked and interpolated over. From the integrated \ha-line, we estimate
% SFR = $18 \pm 3$ M$_\odot$ yr$^{-1}$.

%\begin{figure}
%	\centering
%	\includegraphics[width=9cm]{figures/OIII_img.pdf}
%	\caption{2D-image of the \oiii$\lambda$5007 emission line. The location of a bright skyline is marked by the blue box. The location of the emission line is indicated with the red ellipse. Because the host is observed in nodding-mode, negative images of the emission line appear on both sides in the spatial direction.}
%	\label{fig:line}
%\end{figure}

%A measure of the optical-to-X-ray flux ratio is parametrized in terms of the
% "darkness"-parameter $\beta_{OX}$\citep{Jakobsson2004}. Using the optical
% afterglow limits\citep{Cucchiara2011, Cenko2011}, the X-ray lightcurve can be
% interpolated and evaluated at the time of the non-detection. We find
% $\beta_{OX} < 0.79$, consistent with what was reported in \citet{Sakamoto2013}.

%Because the projected distance does not change much between redshift 1.3 and
% 2.1, all conclusions of \citet{Margutti2012} and \citet{Sakamoto2013} relating
% to host offset are unaffected.

\subsection{Imaging observations and SED analysis} \label{SED}

In addition to the spectroscopy presented above, we imaged the field of
GRB~111117A in multiple broad-band filters using the VLT equipped with FORS2
($gRIz$ filters) and HAWK-I ($JHK_{\mathrm{s}}$ filters), long after the burst
has faded. These new data are complemented by a re-analysis of some of the
imagery used in \citet{Margutti2012} and \citet{Sakamoto2013} that are available
to us here (GTC $gri$-band, TNG $R$-band, and Gemini $z$-band). A log of the
photometric observations and measured brightnesses is given in
Table~\ref{tab:phot_overview}.

All data were reduced, analyzed and fitted in a similar manner as described in detail in
\citet{Kruhler2011a} and, more recently, in \citet{Schulze2016}. Very briefly, we use our own
\texttt{python} and IRAF routines to perform a standard reduction which includes
bias/flat-field correction, de-fringing (if necessary), sky-subtraction, and
stacking of individual images. The photometry of the host was tied against magnitudes 
of field stars from the SDSS and 2MASS catalogs in the case of $grizJHK_{\mathrm{s}}$ filters. 
For the $R$ and $I$-band photometry we used the color transformations of
Lupton\footnote{\url{https://www.sdss3.org/dr8/algorithms/sdssUBVRITransform.php}}. 
We convert all magnitudes into the AB system if necessary, and correct for the Galactic 
foreground of $E_{B-V}=0.027~\mathrm{mag}$.

The multi-color spectral energy distribution (SED) is fit by \citet{Bruzual2003}
single stellar population models based on a \citet{Chabrier2003} initial mass
function in \emph{LePhare} \citep{Ilbert2006}, where the redshift is fixed to
the spectroscopic value of $z=2.211$. The best model is obtained with an
unreddened galaxy template, and returns physical parameters of absolute
magnitude ($M_B=-22.0\pm0.1$\,mag), stellar mass ($\log(M_{\star}/M_\odot) =
9.9\pm0.2$), stellar population age ($\tau = 0.5_{-0.3}^{+0.5}$ Gyr) and
star-formation rate ($SFR_{\mathrm{SED}}=11_{-4}^{+9}
M_\odot\,\mathrm{yr}^{-1}$). We show the SED fit in Fig.~\ref{fig:SED}.

Noteworthy is the discrepancy of our new VLT/FORS2 photometry and the
re-analysis of Gemini data to the $z$-band measurements of \citet{Margutti2012} and
\citet{Sakamoto2013}. Both of these authors report $z$-band photometry that is brighter 
by 0.8~mag to 1.0 mag than our value, whereas data taken in bluer filters are in excellent agreement. In fact, the large $i-z$ color is mistakenly interpreted as a 4000\,\AA\,break
driving the galaxy photometric redshift of the earlier works. Using
the revised photometry from Table~\ref{tab:phot_overview}, the photometric redshift of the
galaxy is $z_{\mathrm{phot}}=2.04_{-0.21}^{+0.19}$, consistent with the
spectroscopic value at the 1~$\sigma$ confidence level.

%%%%%%%%%%%%%%%%%%%%%%%%%%%%%%%%%%%%%%%%%%%%%%%%%%%%%%%%%%%%%%%%%%%%%%%%%%%%
%\subsection{Redshift, line-fluxes, SFR, and Mass}
%%%%%%%%%%%%%%%%%%%%%%%%%%%%%%%%%%%%%%%%%%%%%%%%%%%%%%%%%%%%%%%%%%%%%%%%%%%%

%%%%%%%%%%%%%%%%%%%%%%%%%%%%%%%%%%%%%%%%%%%%%%%%%%%%%%%%%%%%%%%%%%%%%%%%%%%%
\subsection{X-ray temporal and spectral analysis}
%%%%%%%%%%%%%%%%%%%%%%%%%%%%%%%%%%%%%%%%%%%%%%%%%%%%%%%%%%%%%%%%%%%%%%%%%%%%

We retrieved the automated data products provided by the \textit{Swift}-XRT GRB
repository\footnote{\url{http://www.swift.ac.uk/xrt\_products/00507901}}
\citep{Evans2009}. The X-ray afterglow light curve can be fit with a single power-law decay with
index $\alpha=1.27_{-0.10}^{+0.12}$. We performed a time-integrated spectral
analysis using data in photon counting (PC) mode in the widest time epoch where
the $0.3-1.5\,\mathrm{keV}$ to $1.5-10\,\mathrm{keV}$ hardness ratio is constant
(namely, from $t-T_0 = 205$~s to $t-T_0 = 203.5$~ks, for a total of 29.1~ks of
data) to prevent spectral changes that can affect the X-ray column density
determination. The obtained spectrum is well described by an absorbed power-law
model. The best-fit spectral parameters are a photon index of $2.1 \pm 0.4$ and
an intrinsic hydrogen column density $N_{\mathrm{H}}$ of $2.4_{-1.6}^{+2.4}
\times 10^{22}$~cm$^{-2}$ ($z=2.211$), assuming a Galactic $N_{\mathrm{H}}$ in
the burst direction of $4.1 \times 10^{20}$~cm$^{-2}$ \citep{Willingale2013}.




%%%%%%%%%%%%%%%%%%%%%%%%%%%%%%%%%%%%%%%%%%%%%%%%%%%%%%%%%%%%%%%%%%%%%%%%%%%%
\section{Reinterpretation of restframe properties}
%%%%%%%%%%%%%%%%%%%%%%%%%%%%%%%%%%%%%%%%%%%%%%%%%%%%%%%%%%%%%%%%%%%%%%%%%%%%


%%%%%%%%%%%%%%%%%%%%%%%%%%%%%%%%%%%%%%%%%%%%%%%%%%%%%%%%%%%%%%%%%%%%%%%%%%%%
\subsection{Classification}
%%%%%%%%%%%%%%%%%%%%%%%%%%%%%%%%%%%%%%%%%%%%%%%%%%%%%%%%%%%%%%%%%%%%%%%%%%%%

As already pointed out \citep{Margutti2012, Sakamoto2013}, GRB~111117A securely 
belongs the short class of GRBs. Because the observed classification
indicators, $T_{90}$ and hardness ratio, do not depend strongly on redshift
\citep{Littlejohns2013a}, the updated redshift does not change this designation
significantly. 
%The intrinsic luminosity increases, as visible in the X-ray light curve
%in Fig. \ref{fig:sxray_lightcurve}, but it still is sub-luminous compared to
%the majority of long GRBs. 
\citet{Bromberg2013} investigated the degree to which the
long and short population distributions overlap and quantified the certainty in
the class membership. GRB~111117A has $96_{-5}^{+3}$ percent probability of
being short. Compared to the other two sGRB candidates at high redshift, GRB~060121
\citep{DeUgartePostigo2006, Levan2006} at $1.7 \lesssim z \lesssim 4.5$
($17_{-15}^{+14}$ per cent) and GRB~090426 \citep{Antonelli2009, Levesque2010,
	Thone2011} at $z = 2.609$ ($10_{-10}^{+15}$ per cent), the certainty in class
membership for GRB~111117A is much higher.

Additionally, \citet{Horvath2010} classifies both GRB~060121 and GRB~090426 as
intermediate duration bursts. This comes from both events having very soft
spectra, as compared to the hard ones typically seen in sGRBs. Intermediate
bursts are very clearly related in their properties to long GRBs
\citep{DeUgartePostigo2011}, so they are unlikely to come from compact object
mergers.

%\begin{figure}
%	\centering
%	\includegraphics[width=9cm]{figures/XLC_111117A_rest.pdf}
%	\caption{Restframe XRT lightcurve of GRB~111117A, compared to the general population of XRT lightcurves of GRBs. The grey shared region is a compilation of long GRB lightcurves where the color represents density and the light blue is other short GRB lightcurves for which redshifts has been determined. Despite the remarkably high redshift, the luminosity is comparable to the bulk of the short burst population and still subluminous compared to the lGRB population.}
%	\label{fig:sxray_lightcurve}
%\end{figure}

%%%%%%%%%%%%%%%%%%%%%%%%%%%%%%%%%%%%%%%%%%%%%%%%%%%%%%%%%%%%%%%%%%%%%%%%%%%%
\subsection{Restframe $N_\mathrm{H}$} \label{restnH}
%%%%%%%%%%%%%%%%%%%%%%%%%%%%%%%%%%%%%%%%%%%%%%%%%%%%%%%%%%%%%%%%%%%%%%%%%%%%

We plot the recalculated $N_\mathrm{H}$ in Fig.~\ref{fig:NH_z} where we compare with the
distributions of complete samples of both long and short GRBs. The lGRB sample
is from \citet{Arcodia2016} and the sGRB sample is from \citet{DAvanzo2014a}.
17 of the 99 long bursts do not have redshifts and likewise 5 out of 16 for
the short sample. Bursts without redshifts have been excluded for both groups.
GRB~111117A occupies a unique position in Fig.~\ref{fig:NH_z} with the highest
$N_H$ of all short burst. The short sample, excluding GRB~111117A, is located at
low redshifts ($z < 1$) and is found to populate a similar column density
environment to long GRBs at similar redshifts \citep{DAvanzo2014a}. The inferred
hydrogen column for GRB~111117A seems to follow the trend with increasing $N_H$
as a function of redshift as found for the long GRB afterglows \citep{Arcodia2016}.

\begin{figure}
	\centering
	\includegraphics[width=9cm]{figures/NH_z.pdf}
	\caption{Rest frame, X-ray derived hydrogen column densities for GRB~111117A compared to complete samples of both long and short populations. The detections are replaced with contours for clarity and the limits are shown with arrows.}
	\label{fig:NH_z}
\end{figure}

% \begin{figure}
% 	\centering
% 	\includegraphics[width=9cm]{figures/Eiso_z.pdf}
% 	\caption{Isotropic equivalent energy for all the representative samples of both long and short GRBs. The sample is from \citet{Turpin2015} and the short is from \citet{DAvanzo2014a}.}
% 	\label{fig:NH_z}
% \end{figure}

%%%%%%%%%%%%%%%%%%%%%%%%%%%%%%%%%%%%%%%%%%%%%%%%%%%%%%%%%%%%%%%%%%%%%%%%%%%%
\subsection{Host galaxy}
%%%%%%%%%%%%%%%%%%%%%%%%%%%%%%%%%%%%%%%%%%%%%%%%%%%%%%%%%%%%%%%%%%%%%%%%%%%%

%From the clear host association, GRB~111117A does not belong to the hostless class of GRBs \citep{Berger2010a} and because the host exhibits emission lines indicative of a population of relatively young stars, 
As the majority of short GRBs \citep{Fong2013b}, the host of GRB~111117A is a
late-type galaxy and is entirely consistent in terms of stellar mass and stellar
age with the general population of short GRB hosts ($\left\langle M _*
\right\rangle = 10^{10.1} M_{\odot}$ and $\left\langle \tau _* \right\rangle =
0.3 $~Gyr) \citep{Leibler2010}. Being a late-type host, both the stellar mass
and sSFR are entirely within the range expected for the hosts of sGRBs
\citep{Behroozi2014}. The SFR is $\sim$1 order of magnitude higher than the
typical SFR for short GRB hosts galaxies \citep{Berger2014} and more similar to
the SFRs found in the hosts of long GRBs at a corresponding redshift
\citep{Kruhler2015}. The high SFR is partly a selection effect, because a less
star-forming galaxy would exhibit weaker emission lines, thus making the
redshift harder to determine. Additionally, it is natural to expect some
evolution in the hosts of sGRBs with redshift as illustrated in Sect.~\ref{restnH}.


%The detection of \lya~is consistent with the SED-inferred absence of dust, despite
%the moderate stellar age and the high X-ray derived hydrogen column density
%which would suggest the opposite. The centroid of the Ly$\alpha$ emission is
%found to be redshifted by $\sim$~220~km~s$^{-1}$ with respect to systemic, which
%is similar to what is found for long GRB hosts \citep{Milvang-Jensen2012a} where
%the outflow is attributed to star formation.


%%%%%%%%%%%%%%%%%%%%%%%%%%%%%%%%%%%%%%%%%%%%%%%%%%%%%%%%%%%%%%%%%%%%%%%%%%%%%
%\subsection{Comparison to long GRBs at z $\sim$ 2}
%%%%%%%%%%%%%%%%%%%%%%%%%%%%%%%%%%%%%%%%%%%%%%%%%%%%%%%%%%%%%%%%%%%%%%%%%%%%%


%%%%%%%%%%%%%%%%%%%%%%%%%%%%%%%%%%%%%%%%%%%%%%%%%%%%%%%%%%%%%%%%%%%%%%%%%%%%
\section{Implications for redshift distribution of short GRBs}
%%%%%%%%%%%%%%%%%%%%%%%%%%%%%%%%%%%%%%%%%%%%%%%%%%%%%%%%%%%%%%%%%%%%%%%%%%%%

A single sGRB at high redshift does little in terms of constraining the redshift
distribution and therethrough the progenitor channels. In particular, other
short-GRB hosts could be missed because they are intrinsically fainter and thus
this high-$z$ event is only detected due to the brightness of the host.
\citet{Berger2014} compiled a sample of sGRB host luminosities, normalized by
the characteristic galaxy luminosity at their respective redshift,
$L_B/L^{\star}_{B}$. 26 out of 39 hosts (66 per cent) in the sample have
redshifts. To convert the SED-inferred $M_B$ of GRB~111117A to
$L_B/L^{\star}_{B}$, we use the characteristic absolute B-band magnitude of the
Schechter function for the blue galaxies ($U - V < 0.25$) in the redshift window
$2.0 \leq z \leq 2.5$ from \citet{Marchesini2007} and find $L_B/L^{\star}_{B} =
1.2$, which is brighter than 70 per cent of the hosts in \citet{Berger2014} with
measured $L_B/L^{\star}_{B}$. 

If we assume that we are able to obtain emission-line redshifts from hosts with
$R < 25$~mag \citep{Kruhler2012}, then we would have missed around 30 per cent
(8 out of 26 from the sample of \citealt{Berger2014} with measured
$L_B/L^{\star}_{B}$), if they were at the redshift of GRB~111117A. Because the
average SFR of galaxies hosting lGRBs is higher than for galaxies hosting sGRBs,
the fraction of missed burst redshifts is likely higher although the cosmic SFR
evolution could play a role in improving redshift determinability.

%If we additionally assume that the hosts missing redshift in
%\citet{Berger2014} follow the redshift distribution of the hosts \textit{with}
%redshift, this implies that we are \textit{not} missing a dominant fraction of
%sGRB redshift at $z \approx 2$, due to host faintness. 

A fraction of the bursts missing redshift are host-less and is therefore likely
at moderate redshifts \citep{Tunnicliffe2014}, but should some of the remainder
be at high redshift, the missed fraction will increase. If we assume that
\textit{all} the bursts that are missing redshifts in \citet{Berger2014} are at
high-$z$ and missed due to host faintness, 22 out of 39 hosts (55 per cent)
would be missed at $z = 2.211$. This serves as an upper limit on the fraction of
missed burst at high-$z$ and illustrates that we are likely \textit{not} missing a
large fraction of sGRBs redshift at $z \approx 2$, due to host faintness.

The theoretical redshift distribution of sGRBs depends on the type of delay-time
function used to model the progenitor system. The redshift of GRB~111117A puts
constraints on the type of delay-time models suitable for modeling. The
likelihood preferred lognormal time delay models investigated in
\citet{Wanderman2015} predicts a rate of sGRBs at $z = 2.211$, $\sim$ two orders
of magnitude lower compared to peak rate ($z = 0.9$). It is stated in
\citet{Wanderman2015} that this preference depends critically on the absence of
non-collapsar sGRB at $z \gtrsim 1.2$. The redshift of the burst, on the other
hand, is close to the expected peak in sGRB rate calculated for the power law
models \citep{Behroozi2014, Wanderman2015}. 



%In terms of Fig. 9 in \citet{Fong2013a}, the host is indeed  brighter than the average galaxies using the evolving luminosity function of galaxies. \todo{What is the average brightness of a galaxy at z = 2?} 
%\todo{Reformulate paragraph}

%%%%%%%%%%%%%%%%%%%%%%%%%%%%%%%%%%%%%%%%%%%%%%%%%%%%%%%%%%%%%%%%%%%%%%%%%%%
\section{Constraints on the progenitor separation}
%%%%%%%%%%%%%%%%%%%%%%%%%%%%%%%%%%%%%%%%%%%%%%%%%%%%%%%%%%%%%%%%%%%%%%%%%%%

At $z = 2.211$, the age of the universe is 3 Gyr. If the progenitor systems of
sGRBs are the merger of two neutron stars, this sets a hard upper limit to the
coalescence timescale for such a system. In the absence of other mechanisms, the
timescale of the orbital decay of the system is set by the energy loss due to
gravitational waves, which in turn is set by the mass of constituent compact
objects and the separation of the two \citep{Postnov2014}. If we assume that the
formation timescale of the first galaxies is short compared to the time since
the big bang \citep{Richard2011} and if we assume a mass of 1.4~$M_\odot$ for
each of the neutron stars at the time of system formation, this places a hard
upper limit on the initial separation, $a_0$, of the two neutron stars of $a_0 <
3.2~R_\odot$.

If we use the stellar population age from our SED fit, then we obtain a
(softer) limit on the initial separation of $a_0 < 2.1~R_\odot$. However, this
does not account for the possibility there could be an underlying stellar
population of older stars from a previous star-formation epoch. The delay time
between formation and explosion is well accommodated by the models of
\citet{Belczynski2006}, although the longest formation channels are excluded.
This is especially true given the late type nature of the host
\citep{OShaughnessy2008}.

%%%%%%%%%%%%%%%%%%%%%%%%%%%%%%%%%%%%%%%%%%%%%%%%%%%%%%%%%%%%%%%%%%%%%%%%%%%%
\section{Conclusions}
%%%%%%%%%%%%%%%%%%%%%%%%%%%%%%%%%%%%%%%%%%%%%%%%%%%%%%%%%%%%%%%%%%%%%%%%%%%%

In this \emph{Letter}, we have provided a revised, spectroscopic redshift for
the short GRB~111117A based on emission lines setting it at $z = 2.211$. This
value supersedes the previous photometric redshift of \citet{Margutti2012} and
\citet{Sakamoto2013}. %The erroneous redshift estimate of previous authors is
attributed to a discrepancy in the measured $z$-band magnitude.

The rest-frame parameters of the burst and the conditions of the burst
environment have been recalculated using the new distance. The X-ray derived
hydrogen column density towards GRB~111117A is the highest within a complete
sample of short hosts and is indicative of an evolution with redshift as found
for the hosts of long GRBs.

The SFR of the host is in the upper end of the sGRB host SFR distribution and
this helped us to measure the redshift for this galaxy. Despite the moderate age
and high $N_\mathrm{H}$, almost no dust is present.

Although a single burst carries little leverage in terms of constraining the
redshift distribution of sGRB, the high redshift of GRB~111117A needs to be
accommodated in progenitor models. A lognormal delay time model predicts a very low volumetric
density of bursts at $z = 2.211$, whereas a power law delay time model peaks
near GRB~111117A. If more sGRBs are at similarly high redshifts, but are missed due to
the faintness of their hosts, a lognormal delay time model will be disfavored.
Compared to a sample of short GRB hosts, GRB~111117A is more luminous than 70 per
cent of the sample with measured luminosities. Conservatively, for 55 per
cent of the sGRB hosts, we would be unable to determine a redshift should they be at 
a similar redshift as GRB~111117A. This implies that we are \textit{not} missing a large
fraction of the sGRBs at $z \sim 2$.

Using the age of the universe at the time of explosion allows us to set
constraints on the maximal separation between the engine constituents at the
time of formation. We find that the maximal separation for two neutron stars at
formation time is $a_0 < 3.2~R_\odot$, which excludes some of the formation
channels with the longest timescales.

All data, code and calculation related to the paper along with the
paper itself are available at \url{https://github.com/jselsing/GRB111117A}.

\begin{acknowledgements}
TK acknowledges support through the Sofja Kovalevskaja Award to P. Schady. 
CT acknowledges support from a Spanish National Research Grant of Excellence under project AYA 2014-58381-P and funding associated to a Ramón y Cajál fellowship under grant number RyC-2012-09984.
AdUP acknowledges support from a Ramón y Cajal fellowship, a BBVA Foundation Grant for Researchers and Cultural Creators,and the Spanish Ministry of Economy and Competitiveness through project AYA2014-58381-P.
Partly based on observations made with the Gran Telescopio Canarias (GTC).


\end{acknowledgements}

\bibliographystyle{mnras}
\bibliography{GRB111117A}

\newpage
%\appendix

\begin{table*}[!ht]

	\centering
	\caption{Overview of the spectroscopic observations. ``JH'' in the slit width refers to observations where a $K$-band blocking filter has been used. The seeing is determined from the width of the spectral trace of a telluric standard star, observed close in time to the host observation. The spectral resolution, $R$, is measured from unresolved telluric absorption lines in the spectrum of the telluric standard star. \label{tab:spec_overview}}
	\begin{tabular}{cccccccc}
		\hline\hline
		{Observation epoch} &  \multicolumn{3}{c}{Exposure time (s)} & Slit width & Airmass & Seeing & $R$   \\ [1.5pt]
        \hline
		(UT) & UVB  & VIS & NIR &  (arcsec)   & {} & (arcsec)  & {VIS/NIR}  \\ [1.5pt]
		\hline
		2011-11-19 01:33 & 2 $\times$ 2400 & 2 $\times$ 2400 & 8 $\times$ 600 & 1.0/1.0/0.9 & 1.49 & 0.75 & 11600/6700 \\
        2013-07-15 09:02 & 2 $\times$ 1200 & 2 $\times$ 1200 & 8 $\times$ 300 & 1.0/1.0/0.9JH & 1.53 & 0.98 & 9600/8900 \\
        2013-08-03 07:37 & 2 $\times$ 1200 & 2 $\times$ 1200 & 8 $\times$ 300 & 1.0/1.0/0.9JH & 1.55 & 0.85 & 11400/11300 \\
        2013-08-03 08:34 & 2 $\times$ 1200 & 2 $\times$ 1200 & 8 $\times$ 300 & 1.0/1.0/0.9JH & 1.49 & 0.85 & 11400/11300 \\
		
		\hline\noalign{\smallskip}
		
	\end{tabular}

\end{table*}

\begin{table*}[!ht]

	\centering
	\caption{Overview of the photometric observations. \label{tab:phot_overview}}
	\begin{tabular}{ccccccc}
		\hline\hline
{Observation epoch} &  Exptime & Telescope/instrument & Filter & Airmass & Image quality & Host brightness\tablefootmark{a}  \\ [1.5pt]
        \hline
({UT}) & ({ks}) &    & {} & & (arcsec)  & (AB mag)  \\ [1.5pt]
		\hline
2013-08-30 07:43 & 1.45 & VLT/FORS2 & $g$ & 1.55 & 0.99 & $24.08\pm 0.09$ \\
2011-11-17 20:07 & 0.80 & GTC/OSIRIS & $g$ & 1.15 & 1.67 & $24.13\pm 0.09$ \\
2011-11-17 20:07 & 1.20 & GTC/OSIRIS & $r$ & 1.11 & 1.50 & $23.93\pm 0.08$ \\        
2013-07-17 08:37 & 1.45 & VLT/FORS2 & $R$ & 1.56 & 0.74 & $23.95\pm 0.06$ \\   
2011-11-28 21:10 & 3.60 & TNG/DOLORES & $R$ & 1.01 & 1.08 & $23.96\pm 0.13$ \\           
2011-11-17 20:07 & 0.36 & GTC/OSIRIS & $i$ & 1.08 & 1.50 & $23.89\pm 0.23$ \\   
2013-08-03 09:23 & 1.35 & VLT/FORS2 & $I$ & 1.54 & 0.93 & $24.22\pm 0.15$ \\           
2011-11-28 06:14 & 1.80 & Gemini/GMOS-N & $z$ & 1.01 & 0.84 & $24.24\pm 0.47$ \\  
2013-07-13 09:33 & 1.08 & VLT/FORS2 & $z$ & 1.49 & 0.63 & $23.76\pm 0.21$ \\             
2013-06-24 09:14 & 1.98 & VLT/HAWK-I & $J$ & 1.70 & 0.63 & $23.13\pm 0.18$ \\        
2013-06-27 09:21 & 1.68 & VLT/HAWK-I & $H$ & 1.63 & 0.91 & $22.94\pm 0.29$ \\   
2013-06-28 09:14 & 1.92 & VLT/HAWK-I & $K_\mathrm{s}$ & 1.65 & 0.76 & $23.07\pm 0.32$ \\   
\hline\noalign{\smallskip}
		
\end{tabular}

\tablefoot{
\tablefoottext{a}{All magnitudes are given in the AB system and are not corrected for the expected Galactic foreground extinction corresponding to a reddening of $E_{B-V}=0.027$\,mag.}}
\end{table*}

 \begin{figure*}
 	\centering
 	\includegraphics[width=16cm]{figures/SEDspecphot.pdf}
 	\caption{SED fit showning the best-fist SED to the derived photometry. The detection of \lya~ is predicted from the SED fit and confirmed by the spectroscopic observations. Overplotted in grey is the observed spectrum. The reason for the spectral gaps at 5500 \AA~and 10000 \AA~is from the merging of the arms.}
 	\label{fig:SED}
 \end{figure*}



\end{document}


